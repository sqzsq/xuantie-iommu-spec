\section{Fault Reporting}
\label{sec:fault}

\subsection{Operation Mode}

The IOMMU operates in one of two modes of error reporting. In \textit{abort-mode}, the
IOMMU terminates the translation and reports the fault to software. It is configurable for
each device whether and how the IOMMU notifies the device that initiated a faulting
transaction.

In \textit{pause-mode}, the IOMMU maintains the state of the faulting transaction before
reporting the fault to software. After handling the fault, the software has the option to
terminate the transaction or resume from the point where the fault occurred.

The mode in which the IOMMU operates in depends on the translation step that the IOMMU is
currently performing. During device table walk, regardless of translation stage, the IOMMU
always operates in abort-mode, albeit the IOMMU still reports the stage during which the
fault is raised. During the address table walk, the IOMMU operates
according to the configuration in the translation descriptor for that device, as specified
in Section \ref{sec:trans_desc}.

\note The device table entries are system-wide configurations, usually set up by the
kernel or the hypervisor. Faults during the device table walk may indicates system-wide
issues, such issues are usually not fixed in a dynamic manner. In contrast, the address
tables are much more dynamic structures, pause-mode allows runtime adjustment to them.
\noteend

In pause-mode, the IOMMU pauses any translation that raises a fault. The in-flight
transaction remains in the state of \textit{in-progress}. The software is required to
resolved the fault and notify the IOMMU that a particular transaction can move forward.
The software has the option to terminate or retry the transaction.

In pause-mode, the IOMMU maintains state information about in-progress transactions for an
\textit{implementation defined} number of devices.  The behavior of the IOMMU when the
storage space for the state information is used up is \textit{implementation defined}. It
is also up to the implementation to specify whether more than one paused transaction
simultaneously is allowed.

\subsubsection{Fault Reporting with One Translation Stages}

When the IOMMU is only performing one stage of translation, it operates in abort-mode when
walking the device tables. It works in the mode configured by the translation descriptor
when walking the address tables.

\subsubsection{Fault Reporting with Two Translation Stages}

When the IOMMU is performing two stages of translation, the behavior when the IOMMU is
walking in the stage-2 translation structures, i.e., the host device tables and the
stage-2 address tables, is the same as when the IOMMU is only performing one stage of
translation. When the IOMMU is walking the stage-1 structures, the default behavior is the
same as the stage-2 behavior, however, the host can override the behavior in the
translation descriptor.


\subsection{Fault Reporting Registers}

The IOMMU generates an interrupt when a fault is raised during the translation. A command
MMIO range register \iommuinten\ is provided to enable the interrupt from the IOMMU. When
the E bit is one, the interrupt is enabled. When the register is zero, the interrupt is
suppressed. The format of the \iommuinten\ register is shown in Figure
\ref{fig:iommuinten}.

\begin{figure}[h!t]
    \begin{center}
        \begin{tabular}{@{}Jc@{}I}
    \instbitrange{63}{1} &
    \instbit{0} \\
    \hline
    \multicolumn{1}{|c|}{Reserved} &
    \multicolumn{1}{c| }{E} \\
    \hline
    63 & 1 \\

    \end{tabular}
    \end{center}

    \caption{The \iommuinten\ Register}
    \label{fig:iommuinten}
\end{figure}


\note The capability to control the interrupt is for batch process of possible multiple
faults that occurs at the same time. The handler can suppress the interrupt during the
handling, and checks if there are other faults before returning, a known technique. \noteend

The interrupt number is provided in the register MMIO range register called \iommuintno.
The hardware may choose to hardwire this register. In the case that the platform is
equipped with an AIA interrupt controller, this interrupt number is the minor entity
assigned to the IOMMU in any way that is appropriate. The format of this register is shown
in Figure \ref{fig:iommuintno}.

\begin{figure}[h!t]
    \begin{center}
        \begin{tabular}{@{}Jc}
    \instbitrange{63}{0} \\
    \hline
    \multicolumn{1}{| c | }{IntNo} \\
    \hline
    64 \\

    \end{tabular}
    \end{center}

    \caption{The \iommuintno\ Register}
    \label{fig:iommuintno}
\end{figure}


\note The platform can choose to implement the interrupt with any implementation defined
approach. For example, if AIA is implemented, the IOMMU may generate an MSI to the IMSIC
of a hart. If the platform uses the PLIC, the interrupt can be reported as interrupt
routed through the PLIC. \noteend

The read-only register in the register MMIO range \iommucause\, shown in Figure
\ref{fig:iommucause}, provide the code for the
fault reason. Table \ref{tbl:fault_reasons} provides all the reason codes. 

The RSID field provide the ID of the device that raised the fault. The V bit of \iommucause\
indicates if there is a pending fault that needs to be handled.  A read to \iommucause\
has the side-effect of clearing the V bit if there is no other pending fault.

\begin{figure}[h!t]
    \begin{center}
        \begin{tabular}{@{}I@{}M@{}M@{}S}
    \instbit{63} &
    \instbitrange{62}{RSIDLEN + 16} &
    \instbitrange{RSIDLEN + 15}{16} &
    \instbitrange{15}{0} \\
    \hline
    \multicolumn{1}{|c|}{V} &
    \multicolumn{1}{c|}{Reserved} &
    \multicolumn{1}{c|}{RSID} &
    \multicolumn{1}{c| }{Cause} \\
    \hline
    1 &  & RSIDLEN & 16 \\

    \end{tabular}
    \end{center}

    \caption{The \iommucause\ Register}
    \label{fig:iommucause}
\end{figure}

The read-only register in the register MMIO range \ftval\ provide additional data for the
relevant fault, e.g. the address of the entry that caused the fault. The \ftval\ register
is shown in Figure \ref{fig:ftval}.

\begin{figure}[h!t]
    \begin{center}
        \begin{tabular}{@{}Jc}
    \instbitrange{63}{0} \\
    \hline
    \multicolumn{1}{| c | }{Value} \\
    \hline
    64 \\

    \end{tabular}
    \end{center}

    \caption{The \ftval\ Register}
    \label{fig:ftval}
\end{figure}

Once a request is paused, the IOMMU remains inactive for the device that caused the fault.
It returns failure for all subsequent memory requests from the faulting device until a
write is performed to the command MMIO range register \resume\ with the RSID of the
faulting device. The IOMMU then proceed with the previously-faulted transaction caused by
the device with the written RSID. If a RSID is written, however, there is no such
transaction that previously faulted, the IOMMU ignores the write.

\begin{figure}[h!t]
    \begin{center}
        \begin{tabular}{@{}I@{}M@{}N}
    \instbit{63} &
    \instbitrange{62}{RSIDLEN} &
    \instbitrange{RSIDLEN - 1}{0} \\
    \hline
    \multicolumn{1}{|c|}{T} &
    \multicolumn{1}{c|}{Reserved} &
    \multicolumn{1}{c|}{RSID} \\
    \hline
    1 & & RSIDLEN \\

    \end{tabular}
    \end{center}

    \caption{The \resume\ Register}
    \label{fig:resume}
\end{figure}


The T bit in the \resume\ register is provided for software to specify the type of
resumption.  A value of zero means to terminate the transaction. A value of one means to
retry the faulting translation. When the IOMMU is notified to terminate the transaction,
it returns an error to the device that initiates the transaction. It is up to the device
to decide if it is necessary to re-initiate the same transaction. When the IOMMU is
notified to resume the transaction, the IOMMU restarts the entire translation process as
if it is a newly received transaction.

\note When a fault occurs, there are usually other device-specific operations needed to
completely handle the fault. It is up to the specific device and the corresponding device
driver to address them. \noteend

\subsection{Direct Delivery of Fault Interrupt to VM}

The IOMMU maybe designed to work with two types of interrupt systems, the legacy PLIC and
the AIA.

When PLIC is used, a separate interrupt is not provided when the two-stage translation is supported.
The reasoning is that if such an interrupt is provided, the hypervisor might attempt to
directly assign the interrupt for reporting stage-one translation faults to a guest VM.
However, currently the \ftval\ register is in the register MMIO range, the hypervisor
needs to move this data to the memory page that emulates the register MMIO range for the
IOMMU for the guest anyway. A trap is inevitable. Therefore an extra interrupt number is
not needed.

When AIA is used, the IOMMU does have the option to directly deliver
virtual interrupts when a fault is encountered during the first stage of translation. It
may do so by implementing the MSI translation tables defined by the AIA and send MSIs
directly to the IMSIC registers designated to the VM.

To minimize number of traps the VM encounters during handling of the interrupt, in
addition to update the \ftval\ register, the IOMMU
also needs to perform a write to the 8-bytes starting of the offset of the \ftval\
register within the register MMIO range inside the register MMIO page. The write provide
the same value that is written into the \ftval\ register. Subsequently, the VM does not
trap when reading from the emulated register MMIO range.

%how about interrupt control? that still traps?


\begin{table}
    \centering
    \begin{tabular}{| p{0.05\linewidth} | p{0.95\linewidth} |}
        \hline
         Code  & Reason \\
        \hline
         1      & Reserved bits are not zero in the device table entries \\
        \hline
         2      & The S bit is set in a device table entry, however, the translation descriptor contains zeros for S2MODE field \\
        \hline
         21     & R bit is not set in the stage-one address table entry for a read transaction \\
        \hline
         22     & W bit is not set in the stage-one address table entry for a write transaction  \\
        \hline
         23     & X bit and R bit are not both set in the stage-one address table entry for an execute transaction \\
        \hline
         24     & U bit is set in the stage-one address table entry for a privileged transaction \\
        \hline
         25     & U bit is not set in the stage-one address table entry for an unprivileged transaction \\
        \hline
         26     & A bit and D bit are not both set in a stage-one address table entry \\
        \hline
         51    & R bit is not set in the stage-two address table entry for a read transaction \\
        \hline
         52    & W bit is not set in the stage-two address table entry for a write transaction \\
        \hline
         53    & X bit and R bit are not both set in the stage-two address table entry for an execute transaction \\
        \hline
         54    & U bit is set in the stage-two address table entries \\
        \hline
    \end{tabular}
    \caption{List of Fault Reasons During Address Table Walk}
    \label{tbl:fault_reasons}
\end{table}

