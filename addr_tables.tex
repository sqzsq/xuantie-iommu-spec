\section{Address Tables}
\label{sec:addr_tbl}

The Address Tables are memory resident structure. They are of the same format as the
MMU page table defined by the RISC-V Privileged Architecture. The IOMMU support
Sv32, Sv39 and Sv48 translation modes. The various page sizes are supported.

Despite of the same format, the IOMMU does not utilize all of the attribute bits in the
table entries. Depending on the features supported, certain bits are ignored by the IOMMU.
The rest of the section provides the details.

The address tables are selected by the look-up in the device tables. The IOMMU walks the
address tables, performing access permission checks along the way and produces a physical
address or raises a fault condition for a particular translation request.

% Sharing of the tables are possible but not without consequences.

\subsection{One-Stage Address Translation}

For one-stage translation, the IOMMU initiate the page table walk from the table pointed
to by the Stage One Control field in the translation descriptor. If the transaction is a read,
the R bit in the leaf address table must be set for the read permission to be granted.
Likewise, if the transaction is a write, both R bit and W bit must be set.

Optionally, if the transaction is of execute type, the X bit must be set in the address
table entries.

\note The exact meaning of the access type is up to the individual device to define. Not
all types of the access are available on a given platform. \noteend

When the originating device distinguishes privileged access and unprivileged access, for
example, an AXI bus that supports distinction between normal or privileged access,
transactions that are marked as privileged are only allowed if the U bit is cleared.
Likewise, transactions that are marked as unprivileged are only allowed if the U bit is
set.

% TODO: not sure about if SUM is needed

If the G bit is set, this translation is global.

The A bit and the D bit must be set for any access to succeed. Otherwise the IOMMU raises a fault.

\note For ease of implementation, the A bit and D bit are required to be set. In future
versions, the capability to set the bits by the IOMMU will be considered. \noteend

\subsection{Two-Stage Address Translation}

When the second stage of translation is enabled, all the addresses in the first stage
translation are treated as guest physical addresses and are subject to translation with
the stage-two address tables.

For a translation to succeed in granting the permissions requested in the translation
request during two-stage translation, the permission requested by the device must be
granted in the leaf address table entry in both stages. If the permission is not granted
in any of the stages, the IOMMU raises a fault.

The memory accesses to the entries in the stage-one address tables are treated as read
accesses, therefore, the entries in the last-level stage-two address table that map the
pages that contain the stage-one address table entries must have the R bit set. Otherwise,
the IOMMU raises a fault.

If privileged transaction is supported, the privileged access is only applicable in the
first stage of the translation. The U bit in the stage-one address table entries should be
setup in the same manner as if there is only one stage of translation. The IOMMU raises a
fault when either the access is privileged but the U bit is set, or the access is
unprivileged but the U bit is not set. The memory accesses for the translation of the
stage-one addresses are always considered unprivileged.  Therefore, the U bit in the
second stage address table entries should be set. Otherwise, the IOMMU raises a fault.

The G bit in the stage-two address table entries is ignored by hardware.

\note It is unlikely that the hypervisor assigns a device to be shared by all virtual
machines. \noteend

The A bit and the D bit must be set for any access to succeed. Otherwise the IOMMU raises a fault.

\subsubsection{Software Emulation}

To emulate an IOMMU for a virtual machine, the hypervisor allocates pages the size of
register MMIO range and map the pages at the standard address inside the GPA space. The
hypervisor then fills the base address of these pages to the stage-one bits field in the
corresponding device table for that originating device assigned to a virtual machine.

Subsequent operation on the emulated register IOMMU range by the virtual machine does not
cause traps.

When the guest invalidates the IOMMU cache after alteration of the tables, the access to
the command MMIO range is trapped by the host. The host performs the invalidation on
behalf of the guest.

