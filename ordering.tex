\section{Ordering Considerations}

Since the IOMMU works in the transparent configuration, it is required to behave as much
as how a wire behave. In addition, the IOMMU conforms to the ordering constraint of the
bus it is interfacing with. For example, AXI bus requires that ``all transactions with a
given ID must ordered, but there is no restriction on the ordering of transactions with
different IDs''. The IOMMU is therefore required to maintain order only within a given
RSID.

This implies that if the IOMMU supports pause-mode of fault reporting and if any of the
transactions with a given ID faults, all subsequent transactions should be suspended until
the fault is handled. This requirement implies that the IOMMU implementation should be
able to buffer all subsequent transactions before the fault is handled. The number of
transactions that need to be buffered is highly implementation dependent, and is left for
each implementation to decide.

Handling the fault may result in canceling the transaction either directly by the IOMMU or
by the software, in this case, the IOMMU returns error for all subsequent transactions
with the same RSID.

