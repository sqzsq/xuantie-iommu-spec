\section{Working with IOPMP}

The IOMMU and the IOPMP complement each other and provide in-depth defense against
malicious DMA request by malicious devices.

The current design of IOMMU has full control over the permission granted to each device.
However,when the device table's V bit is cleared, the device obtains direct access to
physical memory.Furthermore, future version of the IOMMU is expected to work with the
Address Translation Service (ATS) of the PCI-Express. ATS allows devices to mark DMA
requests as 'translated' so that the IOMMU simply passes the requests to the
interconnects. This is a potential security hole because a malicious device can always
mark any DMA requests as 'translated'.

In this case, the IOMMU can pair with the IOPMP which specify which physical address range
is allowed for a given device, even if the request is marked as translated, the IOPMP will
be able to deny the access.

\subsection{Built-in IOPMP}

%TODO
The translation descriptor can be optionally extended with the pmpcfg and pmpaddr field
for that device. They follow the same format as defined in the PMP in the RISC-V
Privileged Architecture. After the translation is complete, the output physical address is
compared with the pmpcfg and pmpaddr entries for further checks. 
