\section{TLB}
\label{sec:tlb}

The translation process involves a number of memory read, which may incur a long latency.
The IOMMU can optionally include a Translation Look-aside Buffer(TLB) to cache
translation table walk results. The number of the entry in the TLB is implementation
defined.

The TLB entries are tagged by the VMID field and the ASID field in the translation
descriptor. If a device is only subject to one stage of translation, the VMID is assumed
to be zero. If a device is only subject to stage two translation, the ASID is assumed to
be zero.

The IOMMU provides the invalidate register \invltlb\ to flush the designated entry in
the TLB. The \invltlb\ register is in the command MMIO range.

A write to the \invltlb\ with the value of the RSID of a device invalidates the cached
content of the table structures for that device.

% Prefetching for lower latency

% TODO: can it cache device table entries?

\subsection{TLB Organization}

The TLB implemented by the IOMMU could be of a single or multiple hierarchy, and could be
a monolithic or distributed configuration. However, from the software's point of view, the
IOMMU does not expose the internal structure of the TLB and behave as if there is no TLB.
It is up to the implementation to ensure that all subcomponents of the TLB operate in a
coherent manner.

\subsection{Other Caching Consideration}

The address translation by IOMMU involves a potentially large number of memory accesses,
which may incur a large amount of delay when the TLB misses. It is recommended to
implement further caching mechanism to minimize the delay during walking the
memory-resident table structures. The implementation is required to maintain the
coherency of the caches.

\note One straightforward design is simply caching the most-recently used table entry.
\noteend

